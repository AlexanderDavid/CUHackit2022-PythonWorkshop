%----------------------------------------------------------------------------------------
%	PACKAGES AND THEMES
%----------------------------------------------------------------------------------------
\documentclass[aspectratio=169,xcolor=dvipsnames]{beamer}
\usetheme{SimplePlus}
\setbeamercovered{transparent=25}
\AtBeginSection[]
{
    \begin{frame}
        \frametitle{Overview}
        \tableofcontents[currentsection]
    \end{frame}
}

\usepackage{hyperref}
\usepackage{fontenc}
% \hypersetup{colorlinks=true,urlcolor=blue}
% \urlstyle{same}
\usepackage{graphicx} % Allows including images
\usepackage{booktabs} % Allows the use of \toprule, \midrule and \bottomrule in tables
\usepackage{listings,textcomp,color}
\lstset{language=Python,upquote=true,
  basicstyle=\ttfamily\tiny,
  numberstyle=\tiny,stepnumber=1,numbersep=5pt, tabsize=2,
  showspaces=false,showstringspaces=false,showtabs=false,
  breaklines=true,breakatwhitespace=true,escapeinside=||,
  keywordstyle=\color{blue!70},stringstyle=\color{green!70!black!70},
  commentstyle=\color{black!80}\it
}

% ----------------------------------------------------------------------------------------
%	TITLE PAGE
%----------------------------------------------------------------------------------------

\title[short title]{CUHackit Python Workshop} % The short title appears at the bottom of every slide, the full title is only on the title page
\subtitle{Guided Tour Through My Favorite Python-isms}

\author[Day]{Alex Day}

\institute[CU] % Your institution as it will appear on the bottom of every slide, may be shorthand to save space
{
  Ph.D. Student\\
  School of Computing\\
  \vspace{10px}
  \href{https://github.com/AlexanderDavid/CUHackit2022-PythonWorkshop}{shorturl.at/lrFH2}
  % Clemson University% Your institution for the title page
}
\titlegraphic{
  \includegraphics[width=0.25\textwidth]{imgs/soc.png}
}
\date{January 29, 2022} % Date, can be changed to a custom date


%----------------------------------------------------------------------------------------
%	PRESENTATION SLIDES
%----------------------------------------------------------------------------------------

\begin{document}

\begin{frame}
    % Print the title page as the first slide
    \titlepage
\end{frame}

\begin{frame}{Overview}
    % Throughout your presentation, if you choose to use \section{} and \subsection{} commands, these will automatically be printed on this slide as an overview of your presentation
    \tableofcontents
\end{frame}

%------------------------------------------------
\section{Introductions}
%------------------------------------------------

\begin{frame}{Alex Day}

  \begin{columns}[c]
    \column{.55\textwidth}
    \begin{itemize}
        \item B.S. Computer Science Clarion in Pennsylvania
        \item Ph.D. Student under Dr. Ioannis Karamouzas
        \item Working on Social Robot Navigation in the Motion Planning Lab in Charleston
        \item Programming in Python for the last 7 years
    \end{itemize}

    \column{.3\textwidth}
    \begin{figure}
        \includegraphics[width=1.0\textwidth]{imgs/profile.png}
    \end{figure}
  \end{columns}
\end{frame}

%------------------------------------------------

\begin{frame}{Audience}
    \begin{itemize}[<+-|+>]
        \item What year are you?
        \item How comfortable are you with programming in general, any language?
        \item How comfortable are you with python?
    \end{itemize}
\end{frame}

%------------------------------------------------

\begin{frame}{Python}
  \begin{columns}[c]
    \column{.55\textwidth}
    \begin{itemize}
        \item First released in 1991 by Guido van Rossum (BDFL)
        \item Python 3 was released in 2008
        \item Python Enhancement Proposals (PEPs)
        \item Scripting language popular in machine learning and for rapid prototyping
        \item Massive repository of third party libraries on PyPi
    \end{itemize}

    \column{.3\textwidth}
    \begin{figure}
        \includegraphics[width=1.0\textwidth]{imgs/python.png}
    \end{figure}
  \end{columns}
\end{frame}

%------------------------------------------------

\begin{frame}{Python}
  \begin{example}
    \lstinputlisting[language=Python]{code/introduction.py}
  \end{example}
\end{frame}

%------------------------------------------------

\begin{frame}{Input and Output}
  \begin{example}
    \lstinputlisting[language=Python]{code/io.py}
  \end{example}
\end{frame}

%------------------------------------------------

\begin{frame}{Conditionals}
  \begin{example}
    \lstinputlisting[language=Python]{code/conditionals.py}
  \end{example}
\end{frame}

%------------------------------------------------

\begin{frame}{Iteration (While)}
  \begin{example}
    \lstinputlisting[language=Python]{code/iteration_while.py}
  \end{example}
\end{frame}

%------------------------------------------------

\begin{frame}{Iteration (For)}
  \begin{example}
    \lstinputlisting[language=Python]{code/iteration_for.py}
  \end{example}
\end{frame}

%------------------------------------------------

\begin{frame}{Functions}
  \begin{example}
    \lstinputlisting[language=Python]{code/functions.py}
  \end{example}
\end{frame}

%------------------------------------------------

\begin{frame}{Classes}
  \begin{example}
    \lstinputlisting[language=Python]{code/classes.py}
  \end{example}
\end{frame}

%------------------------------------------------

\begin{frame}[fragile]{Types?}
  \begin{itemize}
    \item Python is duck typed
    \begin{itemize}
        \item If it walks like a \texttt{float} and looks like a \texttt{float} then it's probably a \texttt{float}
    \end{itemize}
    \begin{example}
        \begin{lstlisting}[language=Python]
word = 'Hello'
word = 7
word = 3.63
word = {}
        \end{lstlisting}
    \end{example}
    \item The lack of type safety has pros and cons
    \item \href{https://docs.python.org/3/library/typing.html}{\texttt{typing}} and \href{http://mypy-lang.org/}{\texttt{mypy}} aim to make python statically typed
  \end{itemize}
\end{frame}

%------------------------------------------------

\begin{frame}[fragile]{Typing in Python}
  \begin{example}
    \lstinputlisting[language=Python]{code/functions_typed.py}
  \end{example}
\end{frame}

\begin{frame}[fragile]{Typing in Python}
  \begin{example}
    \begin{lstlisting}[language=bash, otherkeywords={mypy}]
mypy functions_typed.py
# Success: no issues found in 1 source file
    \end{lstlisting}
  \end{example}
\end{frame}


%------------------------------------------------

%------------------------------------------------
\section{List Comprehension}
%------------------------------------------------

\begin{frame}[fragile]{Lists}
  \begin{itemize}
    \item Lists are reference types, similar to \texttt{ArrayList}s in Java or \texttt{Vector}s in C++
    \item Unlike other languages lists do not have a specific type
  \end{itemize}
  \begin{example}
    \begin{lstlisting}[language=Python]
numbers = [1, 2, 3, 4, 5, 6]

# Indexing
print(numbers[1])
>>> 2

# Slicing
print(numbers[1:])
>>> [2, 3, 4, 5, 6]

wacky = [1, 1.5, "one point five", {}]
    \end{lstlisting}
  \end{example}
\end{frame}

%------------------------------------------------

\begin{frame}{List Comprehension}
  \begin{itemize}
    \item Just a syntactical nicity
    \item Intuitive to read even for a non-programmer
    \begin{example}
      \lstinputlisting[language=Python, lastline=1, firstline=1]{code/comprehensions.py}
    \end{example}
  \end{itemize}
\end{frame}

%------------------------------------------------

\begin{frame}{Generation of Lists}
  \begin{itemize}
    \item Allows terse generation of complicated risk
    \begin{example}
      \lstinputlisting[language=Python, lastline=8, firstline=3]{code/comprehensions.py}
    \end{example}
  \end{itemize}
\end{frame}

%------------------------------------------------

\begin{frame}{Filtering of Lists}
  \begin{itemize}
    \item Allows you to filter lists using a boolean function
    \begin{example}
      \lstinputlisting[language=Python, firstline=10, lastline=22]{code/comprehensions.py}
    \end{example}
  \end{itemize}
\end{frame}

%------------------------------------------------

\begin{frame}{Cool List Comprehensions}
  \begin{itemize}
    \item There are also some pretty neat things you can do
    \begin{example}
      \lstinputlisting[language=Python, firstline=24, lastline=26]{code/comprehensions.py}
      \lstinputlisting[language=Python, firstline=28, lastline=30]{code/comprehensions.py}
    \end{example}
  \end{itemize}
\end{frame}

%------------------------------------------------

\section{Dunder Methods}
%------------------------------------------------

\begin{frame}{Dunder Methods}
  \begin{itemize}
    \item Dunder (\textbf{d}ouble-\textbf{under}score) methods are special methods that can be implemented in objects
    \item Similar to the concept of \texttt{toString} in Java
    \item These methods allow special behavior such as:
      \begin{enumerate}
        \item \texttt{\_\_repr\_\_(self)} controls how the object is shown as a string
        \item \texttt{\_\_getitem\_\_(self, idx)} allows you to implement indexing on your object
        \item \texttt{\_\_contains\_\_(self, item)} allows you to implement the \texttt{in} keyword functionality
      \end{enumerate}
    \item \href{https://docs.python.org/3/reference/datamodel.html\#special-methods}{exhaustive list in the Python documentation}
  \end{itemize}
\end{frame}

%------------------------------------------------

\begin{frame}{\texttt{\_\_repr\_\_}}
  \begin{itemize}
    \item Meant to return a debugging-quality string representing the object
    \item Similar to \texttt{\_\_str\_\_}
    \begin{itemize}
      \item \texttt{\_\_str\_\_} is for a user-facing, ambiguous, incomplete string representation
      \item \texttt{\_\_repr\_\_} is for a developer-facing, unambiguous, complete string representation
    \end{itemize}
  \end{itemize}
  \begin{example}
    \lstinputlisting[language=Python]{code/repr.py}
  \end{example}
\end{frame}

%------------------------------------------------

\begin{frame}{\texttt{\_\_lt\_\_}, \texttt{\_\_gt\_\_}, \texttt{\_\_eq\_\_}, etc.}
  \begin{itemize}
    \item Rich comparison operators which work as you would expect
    \item The signatures all take the same form of \texttt{\_\_lt\_\_(self, other)}
    \item These functions \textit{should} return either \texttt{True} or \texttt{False}
  \end{itemize}
  \begin{example}
    \lstinputlisting[language=Python]{code/lt.py}
  \end{example}
\end{frame}

%------------------------------------------------

\begin{frame}{\texttt{\_\_getitem\_\_}, \texttt{\_\_setitem\_\_}, and \texttt{\_\_delitem\_\_}}
  \begin{itemize}
    \item Allows indexing and indexing related operations like slicing
    \item Primarily implemented on custom container objects like dictionaries or lists
    \item \texttt{\_\_getitem\_\_(self, key)}
      \begin{itemize}
        \item Returns a reference to the object at \texttt{key}
        \item Have to special implement the negative indexes
      \end{itemize}
    \item \texttt{\_\_setitem\_\_(self, key, value)}
      \begin{itemize}
        \item Allows the user to insert a \texttt{value} into a \texttt{key}
        \item Should only be implemented if you can also remove from the container
      \end{itemize}
    \item \texttt{\_\_delitem\_\_(self, key)}
      \begin{itemize}
        \item Allows the user to delete an item at a specific \texttt{key}
      \end{itemize}
  \end{itemize}
\end{frame}

%------------------------------------------------

\begin{frame}{\texttt{\_\_getitem\_\_}, \texttt{\_\_setitem\_\_}, and \texttt{\_\_delitem\_\_}}
  \begin{example}
    \lstinputlisting[language=Python, lastline=6]{code/getitem.py}
  \end{example}
\end{frame}

%------------------------------------------------

\begin{frame}{\texttt{\_\_getitem\_\_}, \texttt{\_\_setitem\_\_}, and \texttt{\_\_delitem\_\_}}
  \begin{example}
    \lstinputlisting[language=Python, firstline=8, lastline=30]{code/getitem.py}
  \end{example}
\end{frame}

%------------------------------------------------

\begin{frame}{\texttt{\_\_getitem\_\_}, \texttt{\_\_setitem\_\_}, and \texttt{\_\_delitem\_\_}}
  \begin{example}
    \lstinputlisting[language=Python, firstline=32]{code/getitem.py}
  \end{example}
\end{frame}

%------------------------------------------------

%------------------------------------------------
\section{Walrus Operator}
%------------------------------------------------

\begin{frame}{Walrus Operator}
    \begin{enumerate}
        \item
        \item Explanation
        \item Example
    \end{enumerate}
\end{frame}

\begin{frame}{Walrus Operator}
  \begin{example}
  \lstinputlisting[language=Python]{code/walrus.py}
  \end{example}
\end{frame}

%----------------------------------------------------------------------------------------

\begin{frame}
    \Huge{\centerline{\textbf{The End}}}
\end{frame}

%----------------------------------------------------------------------------------------

\end{document}
