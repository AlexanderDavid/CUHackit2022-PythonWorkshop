%----------------------------------------------------------------------------------------
%	PACKAGES AND THEMES
%----------------------------------------------------------------------------------------
\documentclass[aspectratio=169,xcolor=dvipsnames]{beamer}
\usetheme{SimplePlus}

\usepackage{hyperref}
\usepackage{graphicx} % Allows including images
\usepackage{booktabs} % Allows the use of \toprule, \midrule and \bottomrule in tables
\usepackage{listings,textcomp,color}
\lstset{language=Python,upquote=true,
  basicstyle=\ttfamily\tiny,
  numberstyle=\tiny,stepnumber=1,numbersep=5pt, tabsize=2,
  showspaces=false,showstringspaces=false,showtabs=false,
  breaklines=true,breakatwhitespace=true,escapeinside=||,
  keywordstyle=\color{blue!70},stringstyle=\color{green!70!black!70},
  commentstyle=\color{black!80}\it
}
% ----------------------------------------------------------------------------------------
%	TITLE PAGE
%----------------------------------------------------------------------------------------

\title[short title]{CUHackit Python Workshop} % The short title appears at the bottom of every slide, the full title is only on the title page
\subtitle{Python-isms, dunders, and list comprehension}

\author[Day]{Alex Day}

\institute[CU] % Your institution as it will appear on the bottom of every slide, may be shorthand to save space
{
  Ph.D. Student\\
  School of Computing\\
  Clemson University% Your institution for the title page
}
\date{\today} % Date, can be changed to a custom date


%----------------------------------------------------------------------------------------
%	PRESENTATION SLIDES
%----------------------------------------------------------------------------------------

\begin{document}

\begin{frame}
    % Print the title page as the first slide
    \titlepage
\end{frame}

\begin{frame}{Overview}
    % Throughout your presentation, if you choose to use \section{} and \subsection{} commands, these will automatically be printed on this slide as an overview of your presentation
    \tableofcontents
\end{frame}

%------------------------------------------------
\section{Introductions}
%------------------------------------------------

\begin{frame}{Alex Day}

  \begin{columns}[c]
    \column{.55\textwidth}
    \begin{itemize}
        \item B.S. Computer Science Clarion in Pennsylvania
        \item Ph.D. Student under Dr. Ioannis Karamouzas
        \item Working on Social Robot Navigation in the Motion Planning Lab in Charleston
        \item Big fan editors that are hard to use (vim, emacs)
        \item Programming in Python for the last 7 years
    \end{itemize}

    \column{.3\textwidth}
    \begin{figure}
        \includegraphics[width=1.0\textwidth]{profile.png}
    \end{figure}
  \end{columns}
\end{frame}

%------------------------------------------------

\begin{frame}{Python}
  \begin{columns}[c]
    \column{.55\textwidth}
    \begin{itemize}
        \item First released in 1991 by Guido van Rossum (BDFL)
        \item Python 3 was released in 2008
        \item Scripting language popular in machine learning and for rapid prototyping
        \item Massive repository of third party libraries on PyPi
    \end{itemize}

    \column{.3\textwidth}
    \begin{figure}
        \includegraphics[width=1.0\textwidth]{python.png}
    \end{figure}
  \end{columns}
\end{frame}

%------------------------------------------------

\begin{frame}{Python}
  \begin{example}
    \lstinputlisting[language=Python]{introduction.py}
  \end{example}
\end{frame}

%------------------------------------------------

\begin{frame}[fragile]{Types?}
  \begin{itemize}
    \item Python has no concept of type safety
    \begin{example}
        \begin{lstlisting}[language=Python]
word = 'Hello'
word = 7
word = 3.63
word = {}
        \end{lstlisting}
    \end{example}
    \item The lack of type safety has pros and cons
  \end{itemize}
\end{frame}

%------------------------------------------------
\section{List Comprehension}
%------------------------------------------------

\begin{frame}[fragile]{Lists}
  \begin{itemize}
    \item Lists are reference types, similar to \texttt{ArrayList}s in Java or \texttt{Vector}s in C++
    \item Unlike other languages lists do not have a specific type
  \end{itemize}
  \begin{example}
    \begin{lstlisting}[language=Python]
numbers = [1, "2", 3, "4", 5, "6"]

# Indexing
print(numbers[1])
numbers[1] = ["hello world"]
>>> "2"

# Slicing
print(numbers[1:])
>>> [["hello world"], 3, "4", 5, "6"]
    \end{lstlisting}
  \end{example}
\end{frame}

%------------------------------------------------

\begin{frame}{List Comprehension}
  \begin{itemize}
    \item
  \end{itemize}
\end{frame}

%------------------------------------------------
\section{Dunder Methods}
%------------------------------------------------

%------------------------------------------------
\section{Walrus Operator}
%------------------------------------------------

\begin{frame}{Walrus Operator}
    \begin{enumerate}
        \item
        \item Explanation
        \item Example
    \end{enumerate}
\end{frame}

\begin{frame}{Walrus Operator}
  \begin{example}
  \lstinputlisting[language=Python]{walrus.py}
  \end{example}
\end{frame}

%----------------------------------------------------------------------------------------

\begin{frame}
    \Huge{\centerline{\textbf{The End}}}
\end{frame}

%----------------------------------------------------------------------------------------

\end{document}
